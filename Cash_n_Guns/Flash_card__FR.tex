%              %
%%            %%
%%% PREAMBLE %%%
%%            %%
%              %


% Document class
\documentclass[11pt]{beamer}

\usetheme{Boadilla}
\usecolortheme{beaver}
\useinnertheme{rectangles}

\setbeamertemplate{navigation symbols}{}

% Font
\usepackage{fontspec}
\setmainfont[%
	SmallCapsFont={* Caps},%	enable small capital font family
	SlantedFont={* Slanted},%	enable slanted font family
]{Latin Modern Roman}

% Language and typography
\usepackage{polyglossia}
\setdefaultlanguage{french}

\usepackage[autostyle=true]{csquotes}

\usepackage{fmtcount}

\usepackage{microtype}

% Defining 'theorem' environments
%\setbeamertemplate{theorems}[numbered]
\setbeamertemplate{theorem begin}
{%
	\normalfont
	\begin{\inserttheoremblockenv}
		{%
			\inserttheoremname
			\inserttheoremnumber
			\ifx\inserttheoremaddition\@empty\else\ : \inserttheoremaddition\fi%
		}%
}%
\setbeamertemplate{theorem end}{\end{\inserttheoremblockenv}}

% References
\usepackage{cleveref}

\newtheorem{variant}{Variante}



%              %
%%            %%
%%% DOCUMENT %%%
%%            %%
%              %

% Information
\newcommand{\thegame}{Cash'n Guns}

\title[\thegame{} : fiches]{\thegame{}: fiches \enquote{flash cards}}
\author[A. Quenon]{Alexandre Quenon}
\date{\today}

% Text
\begin{document}
% *** Title page *** %
\frame{\titlepage}


% *** Overview *** %
\begin{frame}
	\tableofcontents
\end{frame}



% *** Présentation *** %
\section{Présentation}


\subsection{Le jeu de base}

	\begin{frame}
		\frametitle{\thegame{}: en bref}
		
		\structure{Thèmes généraux}:
		\begin{itemize}
			\item gangsters,
			\item jeu d'ambiance.
		\end{itemize}
	
		\vspace*{1ex}
		
		\structure{Mécanismes}:
		\begin{itemize}
			\item bluff,
			\item prise de risque,
			\item gestion de ressources (cartes \emph{Balle} et \emph{Butin}).
		\end{itemize}
	\end{frame}


\subsection{Les extensions}

	\begin{frame}
		\frametitle{More Cash'n More Guns: en bref}
		
		\texttt{Pas encore d'information.}
	\end{frame}



% *** Rules *** %
\section{Règles}


\subsection{Règles de base}

	\begin{frame}
		\frametitle{Début de partie}
		
		\structure{Mise en place}:
		\begin{enumerate}
			\item chaque joueur prend
				\begin{itemize}
					\item 1 pistolet,
					\item 5 cartes \emph{Clic},
					\item 3 cartes \emph{Bang !},
				\end{itemize}
			\item mélanger les cartes \emph{Butin} et les séparer en 8 piles de 8 cartes;
			\item placer la tuile \emph{Nouveau Parrain} au centre de la table;
			\item placer les jetons \emph{Blessure} à portée;
			\item désigner le premier joueur et lui donner le \emph{bureau du Parrain}.
		\end{enumerate}
	
		\vspace*{1ex}
	
		\structure{Astuces}:
		\begin{itemize}
			\item empiler les 8 piles de 8 cartes \emph{Butin} orthogonalement;
			\item proposition de premier joueur par les règles $\rightarrow$ joueur le plus âgé.
		\end{itemize}
	\end{frame}

	\begin{frame}
		\frametitle{Tour de jeu}
		
		\structure{Déroulement d'un tour}:
		\begin{enumerate}
			\item étaler face visible 8 cartes \emph{Butin} et la tuile \emph{Nouveau Parrain} face bureau;
			\item simultanément, chaque joueur choisit une carte \emph{Balle} (soit une carte \emph{Clic}, soit une carte \emph{Bang!}) et la place face cachée devant lui;
			\item simultanément, au compte de 3, chaque joueur en braque un autre avec le pistolet;
			\item le Parrain actuel peut utiliser son privilège pour forcer un joueur à changer librement de cible;
			\item simultanément, au compte de 3, chaque joueur choisit 1) de se coucher, ou 2) de rester debout et de crier \enquote{Banzaï!};
			\item simultanement, appliquer les cartes des joueurs non couchés;
			\item en partant du Parrain, dans le sens horlogique, chaque joueur encore debout choisit une carte du butin, jusqu'à épuisement du butin.
		\end{enumerate}
	\end{frame}

	\begin{frame}
		\frametitle{Tour de jeu}
		
		\structure{Actions des joueurs}:
		\begin{itemize}
			\item un joueur qui se couche
				\begin{itemize}
					\item défausse sa carte \emph{Balle} face cachée,
					\item ne participe pas au partage du butin,
					\item ne peut pas se faire tirer dessus,
				\end{itemize}
			\item un joueur ciblant un joueur couché défausse sa carte \emph{Balle} face cachée;
			\item un joueur ciblé par 1 ou $n$ \emph{Bang !} subit 1 ou $n$ Blessures, prend autant de jetons \emph{Blessure} et doit coucher son Personnage;
			\item un joueur totalisant 3 jetons \emph{Blessure} ou plus est éliminé de la partie.
		\end{itemize}
	
		\structure{Remarques}:
		\begin{itemize}
			\item le Parrain actuel applique les différentes phases du tour (aide-mémoire au dos de la carte \emph{bureau du Parrain});
			\item si un joueur a été trop lent durant la phase de braquage, on considère qu'il ne braque personne durant le tour.
		\end{itemize}
	\end{frame}

	\begin{frame}
		\frametitle{Fin de partie}
		
		\structure{Conditions}:
		\begin{itemize}
			\item il ne reste qu'un seul joueur en vie;
			\item le \ordinalnum{8} tour de jeu se termine.
		\end{itemize}
		
		Rappel: un joueur totalisant 3 jetons \emph{Blessure} ou plus est immédiatement éliminé.
		
		\vspace*{1ex}
		
		\structure{Décompte} du cash pour les joueurs encore en vie:
		\begin{enumerate}
			\item chaque joueur compte le nombre de cartes \emph{Diamant} qu'il possède 
				\begin{itemize}
					\item celui qui en a le plus gagne le bonus de 60000 \$,
					\item en cas d'égalité personne ne remporte le bonus,
				\end{itemize}
			\item chaque joueur compte la valeur de butin
			\begin{itemize}
				\item le joueur le plus riche gagne la partie,
				\item en cas d'égalité le joueur ayant le plus de Blessures gagne la partie,
				\item en cas d'égalité les joueurs partagent la victoire.
			\end{itemize}
		\end{enumerate}
	\end{frame}


\subsection{Variantes}

	\begin{frame}
		\frametitle{Variantes pour pimenter le jeu}
		
		\begin{variant}[Utiliser les cartes \emph{Pouvoir} \label{var--power}]
			En début de partie, distribuer une carte \emph{Pouvoir} à chaque joueur.
			Les Pouvoirs sont permanents.
		\end{variant}
	
		Remarque: je n'ai pas encore testé la \cref{var--power}.
		
		Risque potentiel: déséquilibre dans le jeu.
	\end{frame}

%
\end{document}