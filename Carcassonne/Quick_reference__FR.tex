%              %
%%            %%
%%% PREAMBLE %%%
%%            %%
%              %


% Document class
\documentclass[11pt]{beamer}

\usetheme{Boadilla}
\usecolortheme{beaver}
\useinnertheme{rectangles}

\setbeamertemplate{navigation symbols}{}

% Font
\usepackage{fontspec}
\setmainfont[%
	SmallCapsFont={* Caps},%	enable small capital font family
	SlantedFont={* Slanted},%	enable slanted font family
]{Latin Modern Roman}

% Language and typography
\usepackage{polyglossia}
\setdefaultlanguage{french}

\usepackage[autostyle=true]{csquotes}

\usepackage{microtype}

% Defining 'theorem' environments
%\setbeamertemplate{theorems}[numbered]
\setbeamertemplate{theorem begin}
{%
	\normalfont
	\begin{\inserttheoremblockenv}
		{%
			\inserttheoremname
			\inserttheoremnumber
			\ifx\inserttheoremaddition\@empty\else\ : \inserttheoremaddition\fi%
		}%
}%
\setbeamertemplate{theorem end}{\end{\inserttheoremblockenv}}

\newtheorem{variant}{Variante}



%              %
%%            %%
%%% DOCUMENT %%%
%%            %%
%              %

% Information
\title{Carcassonne: résumé des règles}
\author[A. Quenon]{Alexandre Quenon}
\date{\today}

% Text
\begin{document}
% *** Title page *** %
\frame{\titlepage}


% *** Overview *** %
\begin{frame}
	\tableofcontents
\end{frame}

\AtBeginSection{%
	\begin{frame}
		\tableofcontents[currentsection]
	\end{frame}
}


% *** Introduction *** %
\section{Introduction}

	\subsection{Le jeu de base}

\begin{frame}
	\frametitle{Carcassonne: en bref}
	
	Thèmes généraux:
	\begin{itemize}
		\item Moyen-Âge,
		\item développement/gestion d'un territoire.
	\end{itemize}

	\vspace*{1ex}
	
	Mécanismes:
	\begin{itemize}
		\item placement de tuiles,
		\item contrôle de territoires par majorité,
		\item gestion de ressources (meeple).
	\end{itemize}
\end{frame}


	\subsection{Les extensions}

\begin{frame}
	\frametitle{Extensions: en bref}
	
	Mini-extension: \alert{La Rivière}.
	
	Une rivière façonne le paysage en début de partie.
	
	Caractéristiques:
	\begin{itemize}
		\item variante sur le terrain,
		\item après quelques parties.
	\end{itemize}


	\vspace*{1ex}
	
	
	Mini-extension: \alert{L'Abbé}.
	
	Nouvelle zone: jardin (équivalent abbaye).
	Meeple spécifique: abbé.
	
	Caractéristiques:
	\begin{itemize}
		\item offre davantage de stratégies,
		\item pour joueurs expérimentés.
	\end{itemize}
\end{frame}


% *** Rules *** %
\section{Règles}

	\subsection{Règles de base}

\begin{frame}
	\frametitle{Début de partie}
	
	Préparation:
	\begin{enumerate}
		\item placer la tuile de départ sur le terrain;
		\item choisir une couleur et prendre les huit meeples associés;
		\item mélanger les tuiles et les disposer en plusieurs piles, faces cachées;
		\item placer le plateau de score avec un meeple de chaque joueur debout sur la case \enquote{0}.
	\end{enumerate}

	Remarque: si un jour fait le tour du plateau de score ($>50$), son meeple sera couché pour l'indiquer.
\end{frame}

\begin{frame}
	\frametitle{Tour d'action}
	
	Règles du tour d'action:
	\begin{itemize}
		\item un joueur à la fois (tour par tour);
		\item réaliser un tour d'action complet, puis passer au joueur suivant (sens horlogique).
	\end{itemize}

	\vspace*{1ex}
	
	Déroulement d'un tour:
	\begin{enumerate}
		\item piocher une tuile au hasard (cf. variantes~\ref{var:choose_tile_common} et \ref{var:choose_tile_own});
		\item poser la tuile $\rightarrow$ les zones doivent concorder;
		\item{}[optionnel] poser un meeple (voleur, chevalier, moine, paysan) dans une zone non contrôlée par un autre meeple (quel que soit le joueur à qui il appartient, y compris soi-même);
		\item évaluer les zones complétées $\rightarrow$ voir \hyperlink{frame:score}{\beamerbutton{Marquer des points}};
		\item récupérer les meeples des zones complétées (retour au propriétaire d'origine).
	\end{enumerate}
\end{frame}

\begin{frame}
	\frametitle{Fin de partie}
	
	Condition: fin du tour du joueur piochant la dernière tuile.
	
	\vspace*{1ex}
	
	Déroulement:
	\begin{enumerate}
		\item finir le tour d'action normalement;
		\item décompter les points dûs aux meeples restant;
		\item évaluer le score final.
	\end{enumerate}

	\vspace*{1ex}
	
	Victoire: score le plus élevé.
\end{frame}

\begin{frame}
	\frametitle{Marquer des points}
	\label{frame:score}
	
	Règles générales:
	\begin{itemize}
		\item les points varient s'ils sont attribués en cours ou en fin de partie;
		\item les points sont obtenus par le joueur en majorité sur la zone;
		\item en cas d'égalité, tous les joueurs obtiennent la totalité des points (cf. variante~\ref{var:divide_pts_draw}).
	\end{itemize}

	\vspace*{1ex}
	
	Points par types de meeples/zones:
	\begin{itemize}
		\item voleur $\rightarrow$ route = 1 point par tuile;
		\item chevalier $\rightarrow$ ville =
		\begin{itemize}
			\item en cours de jeu, 2 points par tuile + 2 points par blason,
			\item en fin de partie, 1 point par tuile + 1 point par blason ($\times\frac{1}{2}$);
		\end{itemize}
		\item moine $\rightarrow$ abbaye = 1 point + 1 point par tuile adjacente;
		\item paysan $\rightarrow$ pré = en fin de partie, 3 points pour chaque ville complète adjacente.
	\end{itemize}
\end{frame}


	\subsection{Variantes}

\begin{frame}
	\frametitle{Variantes pour diminuer le hasard}
	
	Expérience de jeu: piocher une tuile au hasard peut être exaspérant voire amener un blocage.
	Pour diminuer l'impact du hasard, la variante~\ref{var:choose_tile_common} ou \ref{var:choose_tile_own} peut être utilisée \cite{JeuxNim_Carcassonne}.
	
	\begin{variant}[Choisir une tuile parmi $N$ (pot commun)\label{var:choose_tile_common}]
		$N$ tuiles sont dévoilée en permanence.
		Le joueur dont s'est le tour en choisit une et en dévoile immédiatement une autre. Puis, il joue normalement.
		
		Suggestions: $N=3$ (Quenon), $N=2$ \cite{JeuxNim_Carcassonne}.
	\end{variant}

	\begin{variant}[Choisir une tuile parmi $N$ (\enquote{main} personnelle)\label{var:choose_tile_own}]
		Au début de la partie, chaque joueur pioche $N$ tuiles.
		Le joueur dont c'est le tour en choisit une et en pioche immédiatement une autre. Puis, il joue normalement.
		
		Suggestion: $N=3$ \cite{JeuxNim_Carcassonne}.
	\end{variant}
\end{frame}

\begin{frame}
	\frametitle{Variantes pour pimenter le jeu}
	
	Expérience de jeu: création de paysages non réalistes comportant des \enquote{trous} (zones vides).
	Pour augmenter le réalisme, la variante~\ref{var:no_hole} peut être mise en place \cite{JeuxNim_Carcassonne}.
	
	\begin{variant}[Interdiction de créer des trous dans le paysage\label{var:no_hole}]
		Lors de la construction, aucune zone non construite ne peut être entourée de tuiles.
		
		Suggestion: après quelques parties.
	\end{variant}

	Expérience de jeu: manque de pugnacité en cas d'égalité de meeples, surtout si $>2$ joueurs.
	Pour augmenter les enjeux, la variante~\ref{var:divide_pts_draw} peut être utilisée (proposition de Laurent Vanstraelen).
	
	\begin{variant}[Division des points en cas d'égalité\label{var:divide_pts_draw}]
		Si $N$ joueurs contrôlent une même zone à égalité, chacun reçoit le nombre total de points divisé par $N$, arrondi à l'entier inférieur (toujours $\geq 1$).
		
		Suggestion: pour joueurs expérimentés.
	\end{variant}
\end{frame}


% *** References *** %
\section{Références}

\begin{frame}
	\frametitle{Références}
	
	\begin{thebibliography}{Biblio}
		\bibitem[Maréchal]{JeuxNim_Carcassonne}
		N.~Maréchal.
		\newblock Carcassonne.
		\newblock Jeux de Nim, \url{https://www.jeuxdenim.be/jeu-Carcassonne}
	\end{thebibliography}
\end{frame}
%
\end{document}