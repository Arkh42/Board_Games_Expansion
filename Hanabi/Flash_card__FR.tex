%              %
%%            %%
%%% PREAMBLE %%%
%%            %%
%              %


% Document class
\documentclass[11pt]{beamer}

\usetheme{Boadilla}
\usecolortheme{beaver}
\useinnertheme{rectangles}

\setbeamertemplate{navigation symbols}{}

% Font
\usepackage{fontspec}
\setmainfont[%
	SmallCapsFont={* Caps},%	enable small capital font family
	SlantedFont={* Slanted},%	enable slanted font family
]{Latin Modern Roman}

% Language and typography
\usepackage{polyglossia}
\setdefaultlanguage{french}

\usepackage[autostyle=true]{csquotes}

\usepackage{fmtcount}

\usepackage{microtype}


% References
\usepackage{cleveref}



%              %
%%            %%
%%% DOCUMENT %%%
%%            %%
%              %

% Information
\newcommand*{\thegame}{Hanabi}

\title[\thegame{} : fiches]{\thegame{}: fiches \enquote{flash cards}}
\author[A. Quenon]{Alexandre Quenon}
\date{\today}

% Text
\begin{document}
% *** Title page *** %
\frame{\titlepage}


% *** Overview *** %
\begin{frame}
	\tableofcontents
\end{frame}



% *** Présentation *** %
\section{Présentation}


\subsection{Le jeu de base}

	\begin{frame}
		\frametitle{\thegame{}: en bref}
		
		\structure{Contexte}
		
		Ensemble contre le jeu, les joueurs collaborent afin de tirer le plus beau feu d'artifice.
		
		
		\vspace*{2ex}
		
		
		\structure{Mécanismes et objectif}
		
		Chaque joueur possède des cartes qu'il ne connaît pas et les tient faces visibles vers les autres joueurs.
		Le but est de constituer 5 séries de couleurs différentes en posant les cartes dans l'ordre croissant, sans doublon.
	\end{frame}


%\subsection{Les extensions}
%
%	\begin{frame}
%		\frametitle{More Cash'n More Guns: en bref}
%		
%		\texttt{Pas encore d'information.}
%	\end{frame}
%
%
%
% *** Rules *** %
\section{Règles}


\subsection{Règles de base}

	\begin{frame}
		\frametitle{Début de partie}
		
		\structure{Mise en place}
		
		\begin{enumerate}
			\item prendre les 8 jetons bleus \enquote{Indice} et les 3 jetons rouges \enquote{Erreur} et les placer à portée (pot commun);
			\item distribuer à chaque joueur
				\begin{itemize}
					\item 5 cartes, pour 2 ou 3 joueurs;
					\item 4 cartes, pour 4 ou 5 joueurs;
				\end{itemize}
			\item \alert{aucun joueur ne peut regarder ses cartes};
			\item chaque joueur positionne ses cartes pour qu'elles soient visibles à tous les autres joueurs à l'exception de lui-même;
			\item le reste des cartes constitue la pioche.
		\end{enumerate}
	
		\vspace*{1ex}
	
		\structure{Commencement}
		
		Le joueur avec les vêtements les plus colorés débute.
		Puis, on joue tour à tour, dans le sens des aiguilles d'une montre.
	\end{frame}


	\begin{frame}
		\frametitle{Fin de partie}
		
		\structure{Conditions}
		
		La partie s'achève si l'une des conditions suivantes est remplies:
		\begin{itemize}
			\item les 3 jetons rouges \enquote{Erreur} ont été utilisés;
			\item les 5 couleurs de feux d'artifice sont complétées;
			\item la dernière carte est piochée.
		\end{itemize}
	
		Dans le dernier cas, chaque joueur joue une dernière fois, puis la partie se termine.

		
		\vspace*{2ex}
		
		
		\structure{Score}
		
		Somme de la carte de plus haute valeur de chacune des couleurs posées.
		
		Chaque couleur compte trois \enquote{1}, deux \enquote{2}, \enquote{3} et \enquote{4}, et un seul \enquote{5}.
	\end{frame}

	
	\begin{frame}
		\frametitle{Tour de jeu}
		
		\structure{Actions des joueurs}
		
		Durant son tour, un joueur peut effectuer une des trois actions suivantes:
		\begin{enumerate}
			\item donner un indice à un autre joueur de son choix en utilisant un jeton bleu;
			\item défausser une de ses cartes pour récupérer un jeton bleu;
			\item jouer une de ses cartes pour débuter ou compléter une série.
		\end{enumerate}
	
		\vspace*{1ex}
	
		L'indice consiste en un nombre de cartes que possède le coéquipier, soit d'une couleur précise (deux \enquote{rouge}, zéro \enquote{blanc}), soit d'un chiffre précis (trois \enquote{1}).
		Celui qui donne l'indice pointe les cartes concernées avec le doigt.
		
		
		\vspace*{1ex}
		
		
		Si une carte est jouée mais qu'elle ne peut débuter ou compléter une série, elle est défaussée et un jeton rouge est utilisé.
		Si la carte est un \enquote{5} qui termine une couleur, un jeton bleu est récupéré (bonus).
	\end{frame}



\subsection{Variantes}

	\begin{frame}
		\frametitle{Variantes pour pimenter le jeu}
		
		
		\structure{Chorégraphie minutée}
		
		Lorsqu'un joueur pose une carte, il \alert{peut annoncer la couleur} qu'il va compléter.
		En cas de succès, un jeton bleu est récupéré.
		En cas d'échec, la carte est défaussée et un jeton rouge est utilisé, comme si une erreur avait été commise.
		Le joueur n'est pas obligé d'annoncer la couleur.
		
		
		\vspace*{2ex}
		
		
		\structure{Avalanche de couleurs}
		
		On ajoute les \alert{cartes multicolores} (un seul exemplaire de chaque chiffre), qui compte comme une couleur à part entière.
		Les joueurs doivent donc réaliser six séries de feux d'artifice.
		Deux versions possibles pour les indices:
		\begin{enumerate}
			\item les joueurs peuvent annoncer $x$  \enquote{multicolore} dans la main du coéquipier;
			\item les joueurs n'ont pas le droit d'annoncer \enquote{multicolore}, ces dernières étant considérées simultanément comme \enquote{rouge}, \enquote{jaune}, \enquote{blanc}, \enquote{vert} et \enquote{bleu}.
		\end{enumerate}
	\end{frame}

%
\end{document}