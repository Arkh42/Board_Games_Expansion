%              %
%%            %%
%%% PREAMBLE %%%
%%            %%
%              %


% Document class
\documentclass[11pt]{beamer}

\usetheme{Boadilla}
\usecolortheme{beaver}
\useinnertheme{rectangles}

\setbeamertemplate{navigation symbols}{}

% Font
\usepackage{fontspec}
\setmainfont[%
	SmallCapsFont={* Caps},%	enable small capital font family
	SlantedFont={* Slanted},%	enable slanted font family
]{Latin Modern Roman}

% Language and typography
\usepackage{polyglossia}
\setdefaultlanguage{french}

\usepackage[autostyle=true]{csquotes}

\usepackage{microtype}

\usepackage{fmtcount}

% Defining 'theorem' environments
%\setbeamertemplate{theorems}[numbered]
\setbeamertemplate{theorem begin}
{%
	\normalfont
	\begin{\inserttheoremblockenv}
		{%
			\inserttheoremname
			\inserttheoremnumber
			\ifx\inserttheoremaddition\@empty\else\ : \inserttheoremaddition\fi%
		}%
}%
\setbeamertemplate{theorem end}{\end{\inserttheoremblockenv}}

\newtheorem{variant}{Variante}



%              %
%%            %%
%%% DOCUMENT %%%
%%            %%
%              %

% Information
\title{Nosferatu: résumé des règles}
\author[A. Quenon]{Alexandre Quenon}
\date{\today}

% Text
\begin{document}
% *** Title page *** %
\frame{\titlepage}


% *** Overview *** %
\begin{frame}
	\tableofcontents
\end{frame}

\AtBeginSection{%
	\begin{frame}
		\tableofcontents[currentsection]
	\end{frame}
}


% *** Introduction *** %
\section{Introduction}

	\subsection{Le jeu de base}

\begin{frame}
	\frametitle{Nosferatu: en bref}
	
	Thèmes généraux:
	\begin{itemize}
		\item Vampires,
		\item jeu d'ambiance.
	\end{itemize}

	\vspace*{1ex}
	
	Mécanismes:
	\begin{itemize}
		\item rôles cachés,
		\item bluff.
	\end{itemize}
\end{frame}


	\subsection{Les extensions}
	
\begin{frame}
	\frametitle{Extensions: en bref}

	À ma connaissance, il n'existe aucune extension.
\end{frame}


% *** Rules *** %
\section{Règles}

	\subsection{Règles de base (5 à 8 joueurs)}

\begin{frame}
	\frametitle{Début de partie}
	\label{frame:start}
	
	Préparation ($n$ joueurs):
	\begin{enumerate}
		\item prendre les tuiles correspondant aux rôles (1 Vampire, 1 Renfield et $n-2$ Chasseurs);
		\item choisir un joueur ayant le rôle de Renfield (Maître de Jeu) $\rightarrow$ placer la tuile devant lui face visible;
		\item Renfield $\rightarrow$ attribuer les rôles aux autres joueurs en distribuant les tuiles restantes faces cachées;
		\item constituer la pile \enquote{Horloge} en prenant 1 carte Aurore et $n$ cartes Nuit;
		\item constituer la pile \enquote{Évènements} en mélangeant toutes les cartes restantes;
		\item disposer les 5 tuiles Rituels au centre;
		\item chaque joueur excepté Renfield $\rightarrow$ piocher 2 cartes;
		\item Renfield $\rightarrow$ désigner le premier joueur en posant le Pieu devant lui.
	\end{enumerate}
\end{frame}

%\begin{frame}
%	\frametitle{Tour d'action}
%	
%	Règles du tour d'action:
%	\begin{itemize}
%		\item un joueur à la fois (tour par tour);
%		\item réaliser un tour d'action complet, puis passer au joueur suivant (sens horlogique).
%	\end{itemize}
%
%	\vspace*{1ex}
%	
%	Déroulement d'un tour:
%	\begin{enumerate}
%		\item piocher une tuile au hasard (cf. variantes~\ref{var:choose_tile_common} et \ref{var:choose_tile_own});
%		\item poser la tuile $\rightarrow$ les zones doivent concorder;
%		\item{}[optionnel] poser un meeple (voleur, chevalier, moine, paysan) dans une zone non contrôlée par un autre meeple (quel que soit le joueur à qui il appartient, y compris soi-même);
%		\item évaluer les zones complétées $\rightarrow$ voir \hyperlink{frame:score}{\beamerbutton{Marquer des points}};
%		\item récupérer les meeples des zones complétées (retour au propriétaire d'origine).
%	\end{enumerate}
%\end{frame}

\begin{frame}
	\label{frame:end}
	\frametitle{Fin de partie}
	
	
	Conditions de fin:
	\begin{enumerate}
		\item un joueur tue un autre joueur avec le Pieu,
		\item les 5 Rituels sont accomplis,
		\item 5 Morsures sont distribuées.
	\end{enumerate}
	
	\vspace*{1ex}
	
	Victoire immédiate des Chasseurs:
	\begin{enumerate}
		\item le Vampire est tué;
		\item le \ordinalnum{5} Rituel est achevé.
	\end{enumerate}

	Victoire immédiate du Vampire et de Renfield:
	\begin{enumerate}
		\item un Chasseur est tué;
		\item la \ordinalnum{5}[f] Morsure est distribuée.
	\end{enumerate}
\end{frame}

%\begin{frame}
%	\frametitle{Marquer des points}
%	\label{frame:score}
%	
%	Règles générales:
%	\begin{itemize}
%		\item les points varient s'ils sont attribués en cours ou en fin de partie;
%		\item les points sont obtenus par le joueur en majorité sur la zone;
%		\item en cas d'égalité, tous les joueurs obtiennent la totalité des points (cf. variante~\ref{var:divide_pts_draw}).
%	\end{itemize}
%
%	\vspace*{1ex}
%	
%	Points par types de meeples/zones:
%	\begin{itemize}
%		\item voleur $\rightarrow$ route = 1 point par tuile;
%		\item chevalier $\rightarrow$ ville =
%		\begin{itemize}
%			\item en cours de jeu, 2 points par tuile + 2 points par blason,
%			\item en fin de partie, 1 point par tuile + 1 point par blason ($\times\frac{1}{2}$);
%		\end{itemize}
%		\item moine $\rightarrow$ abbaye = 1 point + 1 point par tuile adjacente;
%		\item paysan $\rightarrow$ pré = en fin de partie, 3 points pour chaque ville complète adjacente.
%	\end{itemize}
%\end{frame}


	\subsection{Variantes (nombre de joueurs)}

\begin{frame}
	\frametitle{De 9 à 10 joueurs}
	
	Préparation ($n$ joueurs):
	\begin{enumerate}
		\item prendre les tuiles correspondant aux rôles (2 Vampires, 1 Renfield et $n-3$ Chasseurs);
		\item la suite est identique aux \hyperlink{frame:start}{\beamerbutton{règles de base}}.
	\end{enumerate}

	\vspace*{1ex}

	Conditions de fin:
	\begin{enumerate}
		\item un joueur tue trois autres joueurs avec le Pieu,
		\item les 5 Rituels sont accomplis,
		\item 8 Morsures sont distribuées,
	\end{enumerate}

	Victoire immédiate des Chasseurs:
	\begin{enumerate}
		\item les deux Vampires sont tués;
		\item le \ordinalnum{5} Rituel est achevé.
	\end{enumerate}
	
	Victoire immédiate des Vampires et de Renfield:
	\begin{enumerate}
		\item deux Chasseurs ou plus sont tués;
		\item la \ordinalnum{8}[f] Morsure est distribuée.
	\end{enumerate}
\end{frame}


\begin{frame}
	\frametitle{À 4 joueurs}
	
	Préparation:
	\begin{enumerate}
		\item prendre les tuiles correspondant aux rôles (1 Vampire et 3 Chasseurs);
		\item distribuer les rôles au hasard à tous les joueurs (tuiles faces cachées);
		\item constituer la pile \enquote{Horloge} en prenant 1 carte Aurore et 5 cartes Nuit;
		\item constituer la pile \enquote{Évènements} en mélangeant toutes les cartes restantes;
		\item disposer les 5 tuiles Rituels au centre;
		\item chaque joueur $\rightarrow$ piocher 2 cartes;
		\item désigner le premier joueur en posant le Pieu devant lui (e.g., joueur qui s'est levé le plus tôt).
	\end{enumerate}

	\vspace*{1ex}

	\hyperlink{frame:end}{\beamerbutton{Conditions de fin et de victoire}} inchangées.
\end{frame}



% *** References *** %
\section{Références}

%\begin{frame}
%	\frametitle{Références}
%	
%	\begin{thebibliography}{Biblio}
%		\bibitem[Maréchal]{JeuxNim_Carcassonne}
%		N.~Maréchal.
%		\newblock Carcassonne.
%		\newblock Jeux de Nim, \url{https://www.jeuxdenim.be/jeu-Carcassonne}
%	\end{thebibliography}
%\end{frame}
%
\end{document}